%% timelock.tex - Algebraic time-lock puzzles
%%
%% Copyright 2014 Jeffrey Finkelstein.
%%
%% This LaTeX markup document is made available under the terms of the Creative
%% Commons Attribution-ShareAlike 4.0 International License,
%% https://creativecommons.org/licenses/by-sa/4.0/.
\documentclass{article}

%% This must come before hyperref.
\usepackage{amsthm}
%% This must come before hyperref.
\usepackage{thmtools}
%% This must come before complexity.
\usepackage{hyperref}
%% This is strongly recommended by biblatex.
\usepackage[english]{babel}
%% This is strongly recommended by biblatex.
\usepackage{csquotes}
\usepackage[backend=biber]{biblatex}
\usepackage{amsmath}
\usepackage{amssymb}
\usepackage{complexity}

\hypersetup{pdftitle={Algebraic time-lock puzzles}, pdfauthor={Jeffrey Finkelstein}}

%\addbibresource{references.bib}

\declaretheorem[numberwithin=section]{theorem}
\declaretheorem[numberlike=theorem]{proposition}
\declaretheorem[numberlike=theorem]{conjecture}
\declaretheorem[numberlike=theorem]{assumption}
\declaretheorem[numberlike=theorem, style=definition]{definition}
\declaretheorem[numberlike=theorem, style=definition]{protocol}

%% \newcommand{\getr}{\overset{R}{\gets}}
\newcommand{\Enc}{\operatorname{Enc}}
\newcommand{\Dec}{\operatorname{Dec}}
\newcommand{\Zp}{\mathbb{Z}^*_p}

\title{Algebraic time-lock puzzles}
\author{Jeffrey Finkelstein}

\begin{document}

\maketitle

%% An \emph{$\NP$ relation} is a set of the form $\{ (x, w) \, | \, (x, w) \in L \}$, where $L \in \P$.
%% An \emph{$\NP$ search function} for an $\NP$ relation $R$ is a function $f$ such that for all $x$, $f(x) = w$ where $(x, w) \in R$ (or if no $w$ exists for a given $x$, then $f(x) = \bot$).

%% A \emph{time-lock puzzle} is a pair $(A, B)$ of Turing machines satisfying the following conditions.
%% \begin{itemize}
%% \item \emph{(efficiency)}
%%   %%  For all inputs $m$, $A(m)$ outputs $c$ in polynomial time.
%%   %%  For all inputs $m$ and all $c$ such that $A(m) = c$, $B(c) = m$
%% \item \emph{(completeness)}
%%   There is a $B$ such that for all $m$, $B(A(m)) = m$.
%% \item \emph{(soundness)}
%%   For all $B$,
%% \end{itemize}

\emph{See also: timed-release encryption, timed commitment schemes}

\section{Time-lock puzzles}

A time-lock puzzle is merely an encryption scheme that has no secret key.
The secret key in an encryption scheme is intended to allow efficient decryption only for the holder of the key.
In a time-lock puzzle, ``efficient'' decryption (in a certain sense) is neither desired nor necessary; any decryption should be ``inefficient'' (that is, should require a large amount of resources).

\begin{definition}
  Suppose $M$ is a finite set of messages.
  A pair of functions $(\Enc, \Dec)$ is a \emph{time-lock puzzle} if $m = \Dec(\Enc(m))$ for all $m \in M$.
\end{definition}

\begin{definition}
  TODO: this is an informal definition of indistinguishability.
  Use the formal definitions from Goldreich, Volume II.

  A time-lock puzzle $(\Enc, \Dec)$ is \emph{secure} if no $\NC$ adversary can distinguish $\Enc(m_1)$ from $\Enc(m_2)$ for any $m_1$ and $m_2$ in $M$ with non-negligible probability.
\end{definition}

\section{Algebraic time-lock puzzles}

\subsection{Greatest common divisor}

\begin{assumption}[Computational GCD assumption]
  For any $\NC$ algorithm $A$, the probability over randomly chosen $k$-bit integers $a$ and $b$ that $A(a, b) = \gcd(a, b)$ is negligible in $k$.
\end{assumption}

\begin{assumption}[Decisional GCD assumption]
  TODO: this is an informal definitions of indistinguishability.

  Let $k$ be the security parameter and $K = 2^k$.
  The following two probability distributions are $\NC$ indistinguishable.
  \begin{itemize}
  \item $(a, b, d)$, where $a$ and $b$ are chosen uniformly at random from $\{1, \dotsc, K\}$ and $d = \gcd(a, b)$
  \item $(a, b, r)$, where $a$, $b$, and $r$ are chosen uniformly at random from $\{1, \dotsc, K\}$
  \end{itemize}
\end{assumption}

\begin{proposition}
  If the decisional GCD assumption is true, then the computational GCD assumption is true.
\end{proposition}
\begin{proof}
  Assume with the intention of producing a contradiction that the computational GCD assumption is false, so there is an $\NC$ algorithm $A(a, b)$ that computes $\gcd(a, b)$ with non-negligible probability (say probability at least $\frac{1}{p(k)}$ for some polynomial $p$ in the security parameter $k$).
  We construct an $\NC$ algorithm $A'$ to distinguish $(a, b, d)$ from $(a, b, r)$ as follows.
  On input $(a, b, z)$, the algorithm runs $A(a, b)$, which outputs an integer $d$.
  Now $A'$ accepts if and only if $z = d$.

  $A'$ is an $\NC$ algorithm since it performs only the computation of $A$, which is itself an $\NC$ algorithm, along with a single comparison of the outputs of $A(a, b)$ with the string $z$.
  The probability that $A'$ accepts $(a, b, d)$ is at least $\frac{1}{p(k)}$.
  The probability that $A'$ accepts $(a, b, z)$, where $z$ is a random string, is the probability that $z = \gcd(a, b)$, which is exactly $\frac{1}{2^k}$.
  The value $\left|\frac{1}{p(k)} - \frac{1}{2^k}\right|$ is non-negligible, and hence $A'$ violates the decisional GCD assumption.
\end{proof}

\begin{protocol}[GCD time-lock puzzle]
  Define time-lock puzzle $(\Enc, \Dec)$ as follows.

  \emph{Encryption:} On input message $m \in \Sigma^k$, output $(mx, my)$, where $x$ and $y$ are coprime integers chosen uniformly at random from $\Sigma^k$. (The products $mx$ and $my$ are elements of $\Sigma^{2k}$.)

  \emph{Decryption:} On input ciphertext $(a, b)$, output $\gcd(a, b)$.
\end{protocol}

\begin{conjecture}
  If the decisional GCD assumption is true, then the GCD time-lock puzzle is secure.
\end{conjecture}

\subsection{Modular inverse}

%% \begin{assumption}[Computational modular inverse assumption]
%%   TODO: should p be random in this assumption?
%%   For any $\NC$ algorithm $A$ and for all $k$-bit primes $p$, the probability over randomly chosen $a$ and $b$ in $\Zp$ that $A(ab, b) \equiv a \pmod{p}$ is negligible in $k$.
%% \end{assumption}

%% \begin{assumption}[Decisional modular inverse assumption]
%%   TODO: this is an informal definitions of indistinguishability.

%%   Let $k$ be the security parameter and $p$ be a $k$-bit prime.
%%   The following two probability distributions are $\NC$ indistinguishable.
%%   \begin{itemize}
%%   \item $(ab, b, a)$, where $a$ and $b$ are chosen uniformly at random from $\Zp$.
%%   \item $(ab, b, r)$, where $a$, $b$, and $r$ are chosen uniformly at random from $\Zp$.
%%   \end{itemize}
%% \end{assumption}

%% \begin{proposition}
%%   If the decisional modular inverse assumption is true, then the computational modular inverse assumption is true.
%% \end{proposition}
%% \begin{proof}
%%   Assume with the intention of producing a contradiction that the computational modular inverse assumption is false, so there is an $\NC$ algorithm $A(ab, b)$ that computes $a$ with non-negligible probability (say probability at least $\frac{1}{p(k)}$ for some polynomial $p$ in the security parameter $k$).
%%   We construct an $\NC$ algorithm $A'$ to distinguish $(ab, b, a)$ from $(ab, b, r)$ as follows.
%%   On input $(ab, b, z)$, the algorithm runs $A(ab, b)$, which outputs an integer $x$.
%%   Now $A'$ accepts if and only if $x = z$.

%%   $A'$ is an $\NC$ algorithm since it performs only the computation of $A$, which is itself an $\NC$ algorithm, along with a single comparison of the outputs of $A(ab, b)$ with the string $z$.
%%   The probability that $A'$ accepts $(ab, b, a)$ is at least $\frac{1}{p(k)}$.
%%   The probability that $A'$ accepts $(ab, b, z)$, where $z$ is a random string, is the probability that $z = a$, which is exactly $\frac{1}{2^k}$.
%%   The value $\left|\frac{1}{p(k)} - \frac{1}{2^k}\right|$ is non-negligible, and hence $A'$ violates the decisional modular inverse assumption.
%% \end{proof}

\begin{protocol}[Modular inverse time-lock puzzle]
  Let $p$ be a $k$-bit prime.
  Define time-lock puzzle $(\Enc, \Dec)$ as follows.

  \emph{Encryption:} On input message $m \in \Zp$, output $(mx, x)$, where $x$ is chosen uniformly at random from $\Zp$.

  \emph{Decryption:} On input ciphertext $(c, d)$, output $cd^{-1}$.
\end{protocol}

\begin{conjecture}
  If the modular inverse assumption (TODO: don't know what this would be) is true, then the modular inverse time-lock puzzle is secure.
\end{conjecture}

\begin{assumption}[Modular square root assumption]
  For any $\NC$ algorithm $A$ and for all $k$-bit primes $p$, the probability over randomly chosen $a$ in $\Zp$ that $A(a^2) \equiv \pm a \pmod{p}$ is negligible in $k$.
\end{assumption}

\begin{conjecture}
  If the modular inverse time-lock puzzle is secure, then the modular square root assumption is true.
\end{conjecture}
%% \begin{proof}
  %% Assume with the intention of producing a contradiction that the modular square root assumption is false, so there is an $\NC$ algorithm $A$ and a prime $p$ such that $A$ computes $a$ from $a^2$ with non-negligible probability in $k$, the length of $p$.
  %% Our goal is to construct an $\NC$ algorithm $A'$ that for some $m_1$ and $m_2$ in $\Zp$ distinguishes $(m_1 x, x)$ from $(m_2 y, y)$, where $x$ and $y$ are chosen uniformly at random from $\Zp$, with non-negligible probability in $k$.
  %% Construct $A'$ as follows.
%% \end{proof}

\subsection{Planar integer linear programming}

This puzzle is a bit more complicated to describe.
The message to encrypt is interpreted as a vector in $\mathbb{Z}^2$.
The ciphertext is a linear program with the message as its optimum solution.

\begin{definition}[Planar integer linear programming]
  Given an $n \times 2$ integer matrix $A$, an $n \times 1$ integer vector $b$, and a $2 \times 1$ integer vector $c$, find a $2 \times 1$ integer vector $x$ such that $Ax \leq b$ and $c^\intercal x$ is maximized.
\end{definition}

\begin{assumption}[Planar integer linear programming assumption]
  For any $\NC$ algorithm $D$, the probability over randomly chosen $A$, $b$, and $c$ that $D(A, b, c)$ outputs $x$ such that $c^\intercal x$ is maximized is negligible in $k$.
\end{assumption}

\begin{protocol}[Planar integer linear programming time-lock puzzle]
  Define time-lock puzzle $(\Enc, \Dec)$ as follows.
  (TODO: need to specify maximum bit size of randomly generated integers.)

  \emph{Encryption:} On input message $(x_1, x_2) \in \mathbb{Z}^2$,
  \begin{enumerate}
  \item \emph{(Random objective function.)}
    Generate random $(c_1, c_2) \in \mathbb{Z}^2$.
  \item \emph{(Force $(x_1, x_2)$ to be vertex with maximum value.)}
    Generate random $(a_{11}, a_{12}), (a_{21}, a_{22}) \in \mathbb{Z}^2$ such that $-\frac{a_{11}}{a_{12}} < -\frac{c_1}{c_2}$ and $-\frac{a_{21}}{a_{22}} > -\frac{c_1}{c_2}$.
    Let $b_1 = a_{11} x_1 + a_{12} x_2$ and $b_2 = a_{21} x_1 + a_{22} x_2$.
  %%%% I don't think a bounded feasible region is necessary. If it is, add this back in. %%%%
  %% \item \emph{(Force the feasible region to be bounded by requiring positive solutions.)}
  %%   Let $a_{31} = -1$, $a_{32} = 0$, and $b_3 = 0$.
  %%   Let $a_{41} = 0$, $a_{42} = -1$, and $b_4 = 0$.
  \item \emph{(Create random feasible region.)}
    For each $i$ in $\{3, \dotsc, n\}$, generate random $(a_{i1}, a_{i2}) \in \mathbb{Z}^2$ and random $b_i \in \mathbb{Z}$ such that $a_{i1} x_1 + a_{i2} x_2 \leq b_i$.
    (TODO: are redundant constraints acceptable?)
  \item Output matrix $A$ whose entries are $a_{ij}$, vector $b$ whose entries are $b_i$, and vector $c$ whose entries are $c_j$.
  \end{enumerate}

  \emph{Decryption:} On input ciphertext $(A, b, c)$, output $x$ such that $Ax \leq b$ and $c^\intercal x$ is maximized.
\end{protocol}

\begin{conjecture}
  If the planar integer linear programming assumption is true, then the planar integer linear programming time-lock puzzle is secure.
\end{conjecture}

We know that there is an $\NC$ algorithm that computes the greatest common divisor of two integers if and only if there is an $\NC$ algorithm that computes an optimal solution to a planar integer linear program.
(TODO: add citation here.)

\begin{conjecture}
  The planar integer linear programming assumption is true if and only if the GCD assumption is true.
\end{conjecture}

\begin{conjecture}
  The planar integer linear programming time-lock puzzle is secure if and only if the GCD time-lock puzzle is secure.
\end{conjecture}

%% \printbibliography

\end{document}
